% This is "sig-alternate.tex" V2.0 May 2012
% This file should be compiled with V2.5 of "sig-alternate.cls" May 2012
%
% This example file demonstrates the use of the 'sig-alternate.cls'
% V2.5 LaTeX2e document class file. It is for those submitting
% articles to ACM Conference Proceedings WHO DO NOT WISH TO
% STRICTLY ADHERE TO THE SIGS (PUBS-BOARD-ENDORSED) STYLE.
% The 'sig-alternate.cls' file will produce a similar-looking,
% albeit, 'tighter' paper resulting in, invariably, fewer pages.
%
% ----------------------------------------------------------------------------------------------------------------
% This .tex file (and associated .cls V2.5) produces:
%       1) The Permission Statement
%       2) The Conference (location) Info information
%       3) The Copyright Line with ACM data
%       4) NO page numbers
%
% as against the acm_proc_article-sp.cls file which
% DOES NOT produce 1) thru' 3) above.
%
% Using 'sig-alternate.cls' you have control, however, from within
% the source .tex file, over both the CopyrightYear
% (defaulted to 200X) and the ACM Copyright Data
% (defaulted to X-XXXXX-XX-X/XX/XX).
% e.g.
% \CopyrightYear{2007} will cause 2007 to appear in the copyright line.
% \crdata{0-12345-67-8/90/12} will cause 0-12345-67-8/90/12 to appear in the copyright line.
%
% ---------------------------------------------------------------------------------------------------------------
% This .tex source is an example which *does* use
% the .bib file (from which the .bbl file % is produced).
% REMEMBER HOWEVER: After having produced the .bbl file,
% and prior to final submission, you *NEED* to 'insert'
% your .bbl file into your source .tex file so as to provide
% ONE 'self-contained' source file.
%
% ================= IF YOU HAVE QUESTIONS =======================
% Questions regarding the SIGS styles, SIGS policies and
% procedures, Conferences etc. should be sent to
% Adrienne Griscti (griscti@acm.org)
%
% Technical questions _only_ to
% Gerald Murray (murray@hq.acm.org)
% ===============================================================
%
% For tracking purposes - this is V2.0 - May 2012

\documentclass{acm_proc_article-me}
\graphicspath{{IMAGES/}}



\usepackage{xspace}
\usepackage{color}

\newcommand{\mypartitle}[2][2.5]{\vspace*{-#1 ex}~\\{\noindent {\bf #2}}}
\newcommand{\mypartitleit}[2][2]{\vspace*{-#1 ex}~\\{\noindent {\it #2}}}
\newcommand{\mypartitletwo}[1]{\vspace*{-3ex}~\\{\noindent \underline{#1}}}

\newcommand{\mycite}[1]{{\cite{#1}}}
\newcommand{\degree}{\ensuremath{^\circ}\xspace}

\newcommand{\mycomment}[1]{{{ \color{red} #1 }}}
%\newcommand{\mycomment}[1]{{{ \uppercase{#1} }}}
%\newcommand{\mycomment}[1]{}

%%%%%%%%%%%%%%%%%%%%%%%%%%%%%%%%%%%%%%%%%%%%%%%
%%% CLUSTERING VARIABLES
%%%%%%%%%%%%%%%%%%%%%%%%%%%%%%%%%%%%%%%%%%%%%%%

\newcommand{\SurfClusterDistance}{\ensuremath{D_{S}}\xspace}
\newcommand{\DCTset}{\ensuremath{\mathbf{O}}\xspace}
\newcommand{\DCTvec}{\ensuremath{\mathbf{o}}\xspace}
\newcommand{\UBMmean}{\ensuremath{\mathbf{m}}\xspace}
\newcommand{\IDmean}{\ensuremath{\mathbf{s}}\xspace}
\newcommand{\IDUBMoffset}{\ensuremath{\mathbf{d}}\xspace}
\newcommand{\Ivector}{\ensuremath{\mathbf{u}}\xspace}
\newcommand{\TVsubspace}{\ensuremath{T}\xspace}
\newcommand{\TVnoise}{\ensuremath{\xi}\xspace}
\newcommand{\TVcova}{\ensuremath{\Sigma_{T\!v}}\xspace}
\newcommand{\TVfirstorder}{\ensuremath{F}\xspace}
\newcommand{\TVzeroorder}{\ensuremath{N}\xspace}
\newcommand{\IvectorDistance}{\ensuremath{D_{T}}\xspace}

\newcommand{\ThreshLocal}{\ensuremath{Th_{1}}\xspace}
\newcommand{\ThreshSurf}{\ensuremath{Th_{2}}\xspace}
\newcommand{\ThreshFinal}{\ensuremath{Th_{3}}\xspace}


%%%%%%%%%%%%%%%%%%%%%%%%%%%%%%%%%%%%%%%%%%%%%%%
%%% TRACKING VARIABLES
%%%%%%%%%%%%%%%%%%%%%%%%%%%%%%%%%%%%%%%%%%%%%%%

%\newcommand{\position}{\ensuremath{\bf{x}}\xspace}
%\newcommand{\position}{\ensuremath{X}\xspace}
\newcommand{\position}{\ensuremath{\mathbf{x}}\xspace}
\newcommand{\displacement}{\ensuremath{\mathbf{d}}\xspace}
\newcommand{\imageposition}{\ensuremath{\mathbf{x}^{im}}\xspace}
\newcommand{\colorhist}{\ensuremath{\mathbf{h}}\xspace}
\newcommand{\partset}{\ensuremath{{\cal P}}\xspace}
%\newcommand{\partset}{\ensuremath{P}\xspace}
\newcommand{\motion}{\ensuremath{\mathbf{v}}\xspace}

\newcommand{\pairset}{\ensuremath{\mathbf{{\cal V}}}\xspace}
\newcommand{\timediff}{\ensuremath{\mathbf{{\Delta}}}\xspace}

\newcommand{\bydef}{\ensuremath{\stackrel{\Delta}{=}}\xspace}


\newcommand{\detlabel}{\ensuremath{l}\xspace}
\newcommand{\labelfield}{\ensuremath{L}\xspace}
\newcommand{\detectionset}{\ensuremath{Y}\xspace}
\newcommand{\detection}{\ensuremath{y}\xspace}
\newcommand{\dettime}{\ensuremath{t}\xspace}
\newcommand{\featurefct}{\ensuremath{S}\xspace}
\newcommand{\confidencescore}{\ensuremath{w}\xspace}
%\newcommand{\confidenceset}{\ensuremath{\mathbf{W}}\xspace}
\newcommand{\confidenceset}{\ensuremath{{\cal W}}\xspace}


\newcommand{\param}{\ensuremath{\lambda}\xspace}
\newcommand{\paramk}{\ensuremath{\param^{r}}\xspace}
\newcommand{\parampos}{\ensuremath{\param^{1}}\xspace}
\newcommand{\paramcol}{\ensuremath{\param^{2}}\xspace}
\newcommand{\npcol}{\ensuremath{\alpha}\xspace}
\newcommand{\npmotion}{\ensuremath{\alpha}\xspace}
\newcommand{\paramkdel}[1]{\ensuremath{\param^{k}_{#1}}\xspace}
\newcommand{\paramposdel}[1]{\ensuremath{\param^{1}_{#1}}\xspace}
\newcommand{\paramcoldel}[1]{\ensuremath{\param^{2}_{#1}}\xspace}

\newcommand{\posset}{\ensuremath{{\mathcal{P}}}\xspace}
\newcommand{\closestset}{\ensuremath{{\mathcal{C}^{\star}}}\xspace}
%\newcommand{\secondset}{\ensuremath{{\mathcal{S}_{\Delta}}}\xspace}
\newcommand{\secondset}{\ensuremath{{\mathcal{S}^{\star}}}\xspace}
\newcommand{\closestsettracklets}{\ensuremath{{\mathcal{C}}}\xspace}
\newcommand{\secondsettracklets}{\ensuremath{{\mathcal{S}}}\xspace}
\newcommand{\colsetzero}{\ensuremath{{\mathcal{C}_{\Delta,H_0}}}\xspace}
\newcommand{\colsetone}{\ensuremath{{\mathcal{C}_{\Delta,H_1}}}\xspace}

\newcommand{\closestsetdeltastar}{\ensuremath{{\mathcal{C}_{\timediff}^{\star}}}\xspace}
\newcommand{\secondsetdeltastar}{\ensuremath{{\mathcal{S}_{\timediff}^{\star}}}\xspace}

\newcommand{\Ngauss}{\ensuremath{N_{mix}}\xspace}


\newcommand{\Ndetections}{\ensuremath{N_{y}}\xspace}
%\newcommand{\Npairwisefct}{\ensuremath{N_2}\xspace}
\newcommand{\Npairwisefct}{\ensuremath{N_{s}}\xspace}
\newcommand{\Nunitaryfct}{\ensuremath{N_1}\xspace}
\newcommand{\Nfeature}{\ensuremath{N_{f}}\xspace}
%\newcommand{\Tshort}{\ensuremath{T_{short}}\xspace}
\newcommand{\Tshort}{\ensuremath{T_{w}}\xspace}

\newcommand{\meanmult}{\ensuremath{{\bf{m}}}\xspace}




\newcommand{\sigmoid}{\ensuremath{{S}}\xspace}

\newcommand{\uniquelabel}{\ensuremath{{\cal U}}\xspace}
\newcommand{\coeflabelfield}{\ensuremath{\rho}\xspace}
\newcommand{\trackcost}{\ensuremath{{C}}\xspace}
\newcommand{\trackcoststart}{\ensuremath{{C}^s}\xspace}
\newcommand{\trackcostend}{\ensuremath{{C}^e}\xspace}
\newcommand{\trackset}{\ensuremath{{\cal \tau}}\xspace}
\newcommand{\tracklet}{\ensuremath{{\tau}}\xspace}
\newcommand{\tstart}{\ensuremath{{t}^s}\xspace}
\newcommand{\tend}{\ensuremath{{t}^e}\xspace}
\newcommand{\timemargin}{\ensuremath{\theta_{tm}}\xspace}
\newcommand{\trackduration}{\ensuremath{{d}}\xspace}
\newcommand{\trackdurationmax}{\ensuremath{{d_{\max}}}\xspace}


\newcommand{\cueconfidence}{\ensuremath{{c}}\xspace}


\newcommand{\parapositionconf}{\ensuremath{{\theta_f}}\xspace}


\newcommand{\assomatrixsl}{\ensuremath{\mathbf{A}^{SW}}\xspace}
\newcommand{\assomatrixbicm}{\ensuremath{\mathbf{A}^{BI}}\xspace}
\newcommand{\Nblockb}{\ensuremath{N^{B}_{\tau}}\xspace}
\newcommand{\Nblocka}{\ensuremath{N^{A}_{\tau}}\xspace}

\newcommand{\labelfieldactive}{\ensuremath{L_a}\xspace}


\endinput

%\newcommand{\mypartitle}[2][2]{\vspace*{-#1 ex}~\\{\noindent {\bf #2}}}

\usepackage{paralist}

\begin{document}
%
% --- Author Metadata here ---
%\conferenceinfo{\textit{MediaEval 2015 Workshop,}}{Sept. 14-15, 2015, Wurzen, Germany}
%\CopyrightYear{2007} % Allows default copyright year (20XX) to be over-ridden - IF NEED BE.
%\crdata{0-12345-67-8/90/01}  % Allows default copyright data (0-89791-88-6/97/05) to be over-ridden - IF NEED BE.
% --- End of Author Metadata ---

\title{(Provisional) Towards large-scaled multimedia indexing: A case-study in person discovery in broadcast news\\[-100mm]}
%
% We need to keep anonymity for the reviewers
%\author{Nam Le$^{1, 2}$, Alexander Heili$^{1}$, Di Wu$^{1}$, Jean-Marc Odobez$^{1, 2}$\\

\author{}
%% \author{Author 1$^{1, 2}$, Author 2$^{1}$, Author 3$^{1, 2}$\\
%% {\footnotesize %$^1$ \affaddr{Idiap Research Institute, Martigny, Switzerland}}\\
%% $^1$ \affaddr{Affiliation address 1}}\\
%% {\footnotesize %$^2$ \affaddr{\'{E}cole Polytechnique F\'{e}d\'{e}ral de Lausanne, Switzerland}}\\
%% $^2$ \affaddr{Affiliation address 2}}\\
%% %{\footnotesize \email{\{nle, aheili, dwu, odobez\}@idiap.ch}}
%% {\footnotesize \email{\{author1, author2, author3\}@mail.com}}
%% }

\maketitle


\begin{abstract}
\vspace*{-1mm}
TBA - Joint paper of Person Discovery consortiumm.
\end{abstract}

\endinput


\section{Introduction}

TV archives maintained by national institutions such as the French INA, the Netherlands Institute for Sound \& Vision, or the British Broadcasting Corporation are rapidly growing in size. The need for applications that make these archives searchable has led researchers to devote concerted effort to developing technologies that create indexes.

Because human nature leads people to be very interested in other people.
Indexes that represent the location and identity of people in the archive are indispensable for searching archives.
%
To this end, started in 2011, the REPERE challenge aimed at supporting research on multimodal person recognition~\cite{BERNARD--SLAM--2013, KAHN--CBMI--2012}. Its main goal was to answer the two questions \emph{``who speaks when?''} and \emph{``who appears when?''} using any available source of information (including pre-existing biometric models and person names extracted from text overlay and speech transcripts). 
%
Thanks to this challenge and the associated multimodal corpus~\cite{GIRAUDEL--LREC--2012}, significant progress was achieved in either supervised or unsupervised multimodal person recognition~\cite{BECHET--INTERSPEECH--2014, BENDRIS--CBMI--2013, BREDIN--ODYSSEY--2014, BREDIN--INTERSPEECH--2013, BREDIN--SLAM--2013, BREDIN--IJMIR--2014, FAVRE--SLAM--2013, GAY--CBMI--2014, POIGNANT--ASLP--2015, POIGNANT--SLAM--2013, POIGNANT--INTERSPEECH--2012, POIGNANT--MTAP--2015, ROUVIER--CBMI--2014}.

However, when the content is created or broadcast, it is not always possible to predict which people will be the most important to find in the future and biometric models may not yet be available at indexing time The goal of this task is thus to address the challenge of indexing people in the archive under real-world conditions, \emph{i.e.} when there is no pre-set list of people to index.
%
This makes the task completely unsupervised (\emph{i.e.} using algorithms not relying on pre-existing labels or biometric models).
%
In order to successfully tag people with the correct identities, one must find a way to assign a name correctly to a presence of the corresponding person, then that name must also be propagated to all the shots during which that person appears and speaks. For this purpose, there are 3 possible approaches:
\begin{compactitem}
\item{Clustering-based name assignment: Face/speech tracks are first aggregated into homogeneous clusters according to person identities. Then each clusters is tagged with the most probable person name.}
\item{Verification-based name propagation: A person name is first assigned to the most probable face/speech track. The name is then propagated to all face/speech tracks which are verified to have the same identity.}
\item{Graph-based name propagation: A graph is built with a face/speech track as a node and weight of edges is the similarity. Some nodes are initially tagged with the names. Names are then propagated along the edges within the graph.}
\end{compactitem}

There are several problems related to these approaches such as face / speech representation, person diarization, or audio-visual verification. Each of these problems has been well studied within its respective context~\cite{recog,veri,rep}. 
%
However, these approaches have never been fully investigated and compared as whole systems in the large-scale multimedia indexing context before. In this paper, the authors aim to investigate all these approaches with variations in their components using real world datasets from TV news. 
%
To emphasize on the unsupervised nature of real world applications, we applied these approaches on a large scale multimedia dataset associated to the ``Multimodal Person Discovery in Broadcast TV'' task~\cite{POIGNANT--MEDIAEVAL--2015,tocite}. In this task, participants are provided with a collection of TV broadcast recordings pre-segmented into shots. Each shot $s \in \shots$ has to be automatically tagged with the names of people both speaking and appearing at the same time during the shot.
%
The benchmarking results allow us to thoroughly analyse all 3 approaches to understand their pros and cons. Thus, we can draw meaningful lessons for good practice in large-scale person discovery in broadcast news.

\endinput


\section{Related Work}
\label{sec:related_work}

TBA

\endinput



\section{Overview}

Given the raw TV broadcasts, each shot must be automatically tagged with the name(s) of people who can be both seen as well as heard in the shot along with the confident score. The list of people is not known apriori and their names must be discovered from video text overlay or speech transcripts~\cite{bredin2016mediaeval}. 
%
To this end, a video must be segmented in an unsupervised way into homogeneous segments according to person identity, like  speaker diarization and face diarization, to be combined with the extracted names.
% . Combined with the extracted names,  audio-visual person diarization makes it possible to identify people in videos. % \cite{Gay:Frontiers:2016}.
%

%\begin{figure}[tb]
%\centering
%\epsfig{file=diagram.png,width=70mm}
%\vspace*{-3mm}
%\caption{Architecture of our system}
%\vspace*{-3mm}
%\label{fig:pipeline}
%\end{figure}

The overall system is illustrated in Fig.~\ref{fig:pipeline}. It consists of  4 main parts: video optical character recognition (OCR) and named entity recognition (NER), face diariation, speaker diarization, and fusion naming. Each of these parts will be described in the following sections.

\endinput


\section{Face clustering}
\label{sec:face_clustering}

TBA

\subsection{First system}

TBA

\subsection{Second system}

TBA

\endinput


\section{Speaker Diarization}
\label{sec:speaker_diarization}

TBA

\endinput


\section{Naming}
\label{sec:naming}

TBA

\subsection{Single modal name propagation}

TBA

\subsection{Multi-modal name propagation}

TBA

\endinput


\section{Experiments}
\label{sec:experiment}
%
\subsection{Datasets and Metric}

The test set is divided into three datasets: INA, DW and 3-24. The INA dataset contains a full week of broadcast for 2 TV channels for a total duration of 90 hours in French. The DW dataset~\cite{EUMSSI} is composed of video downloaded from Deutsche Welle website, in English and German for a total duration of 50 hours. The last dataset contains 13 hours of broadcast from 3/24 Catalan TV news channel. Each shot has been tagged with the names of people who appear and speak within that shot.

The task is evaluated indirectly as an information retrieval task, using the folllowing principle.
%
For each query $q \in \queries \subset \refNames$ (\texttt{first\-name\_lastname}), returned shots are first sorted by the edit distance between the hypothesized person name and the query $q$ and then by confidence scores.
Average precision $\text{AP}(q)$ is then computed classically based on the list of relevant shots (according to the groundtruth) and the sorted list of shots. Finally, Mean Average Precision is computed as follows:
\begin{align}
            \text{MAP} & = \frac{1}{|\queries|} \sum_{q \in \queries} \text{AP}(q) \nonumber
\end{align}

\subsection{Results}

\begin{compactitem}
  \item Sub. (2) used LIUM speaker diarization with our OCR-NER.
  \item Sub. (2) used OpenFace for face naming without talking score with our OCR-NER.
	\item Sub. (3) used our face naming without talking score with our OCR-NER.
	\item Sub. (4) used our face naming with talking score. 
	\item Sub. (5) used the combination of talking face naming in sub. (3) with speaker naming.
\end{compactitem}

\begin{table}[tb]
\centering
\begin{tabular}{c|c|c|c|}
\cline{2-4}
                                & MAP@1  & MAP@10 & MAP@100  \\ \hline
 \multicolumn{1}{|c|}{Sub. (1)} & 29.9   & 26.2   & 25.2 \\ \hline
 \multicolumn{1}{|c|}{Sub. (2)} & 65.8   & 46.0   & 45.0 \\ \hline
 \multicolumn{1}{|c|}{Sub. (3)} & 62.3   & 50.3   & 49.2 \\ \hline
 \multicolumn{1}{|c|}{Sub. (4)} & 69.3   & 57.0   & 55.8 \\ \hline
 \multicolumn{1}{|c|}{Sub. (5)} & 73.6   & 59.8   & 57.9 \\ \hline

\end{tabular}
\vspace*{-2mm}
\caption{Benchmarking results of our submissions. Details of each submission in the text.}
\vspace*{-2mm}
\label{tab:mediaeval}
\end{table}


%BEGINNING results IRISA/PUCMINAS---------------------------------------------------

\begin{table}[!t]
\caption{Mean average precisions @K obtained on the development and test sets using approaches by IRISA - PUC Minas. {\it no prop.}: no tag propagation. MST: tag propagation using the Minimum-Spanning Tree approach. RW: tag propagation using the Random Walk approach.}
\label{table_example}
\centering

\begin{tabular}{|c||c|c|c|c|}

\multicolumn{5}{ c }{Dev16)}\\
\hline
 & \multicolumn{4}{| c |}{MAP@K (\%)}\\
\hline
K  & 1 & 5 & 10 & 100\\
\hline
\hline
no prop.  & 81.4 & 66.3 & 64.2 & 61.8\\
\hline
MST & 81.4 & 71.1 & 69.5 & 67.1\\
\hline
RW & 81.4 & 71.2 & 69.7 & 67.1\\
\hline

\multicolumn{5}{ c }{Test16}\\
\hline
 & \multicolumn{4}{| c |}{MAP@K (\%)}\\
\hline
K  & 1 & 5 & 10& 100\\
\hline
\hline
no prop. & 55.9 & 35.8 & 33.8 & 32.8\\
\hline
MST & 68.9 & 57.5 & 55.4 & 53.6\\
\hline
RW & 71.3 & 59.7 & 57.4 & 55.5\\
\hline
\end{tabular}
\end{table}
%END results IRISA/PUCMINAS---------------------------------------------------

\endinput


\section{Discussion and future works}
\label{sec:discuss}

\endinput


\bibliographystyle{abbrv}
{\scriptsize
\bibliography{PersonDiscovery2016}
}
\end{document}
