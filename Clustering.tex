\section{Clustering-based name assignment}
\label{sec:clustering}

To this end, a video must be segmented in an unsupervised way into homogeneous segments according to person identity, like  speaker diarization and face diarization, to be combined with the extracted names.

\subsection{Face clustering}

Given the video shots, face diarization process consists of (i) face detection, detecting faces appearing within each shot, (ii) face tracking, extending detections into continuous tracks within each shot, and (iii) face clustering, grouping all tracks with the same identity into clusters.

\subsubsection{First system}

Face tracking-by-detection is applied within each shot using a detector based on histogram of oriented gradients~\cite{Dalal2005} and the correlation tracker proposed by \emph{Danelljan et al.}~\cite{Danelljan2014}. Each face track is then described by its average \emph{FaceNet} embedding and compared with all the others using Euclidean distance~\cite{Schroff2015}. Finally, average-link hierarchical agglomerative clustering is applied. Source code for this module is available in \emph{pyannote-video}\footnote{\url{http://pyannote.github.io}}.

\subsubsection{Second system}

A fast version of deformable part-based model (DPM)~\cite{felzenszwalb2010dpm,mathias2014face,dubout2013deformable} is first applied. Then tracking is performed using the CRF-based multi-target tracking framework~\cite{heili2014tracking}, which relies on the unsupervised learning of time sensitive association costs for different features.
%
The detector is only applied 4 times per second and an explicit false alarm classifier at the track level is learned\cite{Le_ICPR_2016}.
%
Each face track is then described using a combination of keypoint matching distances and total variability modeling (TVM)~\cite{wallace2011inter,wallace2012total,Khoury:ICMR:2013}.

\subsection{Speaker diarization}

TBA

\subsection{Name assignment}
Direct assignment + Hungarian algorithm

\endinput
